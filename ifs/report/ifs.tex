\documentclass[a4paper,14pt]{extarticle}

\usepackage[utf8]{inputenc}
\usepackage[T1]{fontenc}
\usepackage[russian]{babel}
\usepackage{hyperref}
\usepackage{indentfirst}
\usepackage{here}
\usepackage{array}
\usepackage{graphicx}
\usepackage{caption}
\usepackage{subcaption}
\usepackage{chngcntr}
\usepackage{amsmath}
\usepackage{amssymb}
\usepackage[left=2cm,right=2cm,top=2cm,bottom=2cm,bindingoffset=0cm]{geometry}
\usepackage{multicol}
\usepackage{multirow}
\usepackage{titlesec}
\usepackage{listings}
\usepackage{color}
\usepackage{enumitem}
\usepackage{cmap}
\usepackage{pgfplotstable}
\usepackage{listingsutf8}

\definecolor{green}{rgb}{0,0.6,0}
\definecolor{gray}{rgb}{0.5,0.5,0.5}
\definecolor{purple}{rgb}{0.58,0,0.82}

\lstdefinelanguage{none}{}

\lstset{
	language={C},
	inputpath={../src},
	backgroundcolor=\color{white},
	commentstyle=\color{green},
	keywordstyle=\color{blue},
	numberstyle=\color{gray}\scriptsize\ttfamily,
	stringstyle=\color{purple},
	basicstyle=\lst@ifdisplaystyle\footnotesize\fi\ttfamily,
	breakatwhitespace=false,
	breaklines=true,
	captionpos=b,
	keepspaces=true,
	numbers=left,
	numbersep=5pt,
	showspaces=false,
	showstringspaces=false,
	showtabs=false,
	tabsize=4,
	frame=single,
	morekeywords={},
	deletekeywords={},
	morecomment=[l][\color{green}]{\$\$},
	sensitive=true,
	columns=fullflexible,
	inputencoding=utf8/cp1251,
	literate=%
		{~}{{\raise.25ex\hbox{$\mathtt{\sim}$}}}{1}%
		{-}{-}{1}
}

\renewcommand{\le}{\ensuremath{\leqslant}}
\renewcommand{\leq}{\ensuremath{\leqslant}}
\renewcommand{\ge}{\ensuremath{\geqslant}}
\renewcommand{\geq}{\ensuremath{\geqslant}}
\renewcommand{\epsilon}{\ensuremath{\varepsilon}}
\renewcommand{\phi}{\ensuremath{\varphi}}
\renewcommand{\thefigure}{\arabic{figure}}
\newcommand{\code}[1]{\lstinline|#1|}
\newcommand{\caret}{\^{}}
\newcommand{\lex}[1]{
	\lstinputlisting[caption=\code{#1.l}]{#1.l}
}
\newcommand{\yacc}[1]{
	\lstinputlisting[caption=\code{#1.y}]{#1.y}
}
\newcommand{\inout}[1]{ 
	\lstinputlisting[language=,caption=\code{#1.in}]{#1.in}
	\lstinputlisting[language=,caption=\code{#1.out}]{#1.out}
}

\titleformat*{\section}{\large\bfseries} 
\titleformat*{\subsection}{\normalsize\bfseries} 
\titleformat*{\subsubsection}{\normalsize\bfseries} 
\titleformat*{\paragraph}{\normalsize\bfseries} 
\titleformat*{\subparagraph}{\normalsize\bfseries} 

\counterwithin{figure}{section}
\counterwithin{equation}{section}
\counterwithin{table}{section}
\newcommand{\sign}[1][5cm]{\makebox[#1]{\hrulefill}}
\newcommand{\equipollence}{\quad\Leftrightarrow\quad}
\newcommand{\no}[1]{\overline{#1}}
\graphicspath{{../pics/}}
\captionsetup{justification=centering,margin=1cm}
\def\arraystretch{1.3}
\setlength\parindent{5ex}
\titlelabel{\thetitle.\quad}

\setitemize{topsep=0em, itemsep=0em}
\setenumerate{topsep=0em, itemsep=0em}

\begin{document}

\begin{titlepage}
\begin{center}
	Санкт-Петербургский Политехнический Университет Петра Великого\\[0.3cm]
	Институт компьютерных наук и технологий \\[0.3cm]
	Кафедра компьютерных систем и программных технологий\\[4cm]
	
	\textbf{ОТЧЕТ}\\ 
	\textbf{по лабораторной работе}\\[0.5cm]
	\textbf{<<Индивидуальное задание на языке Lex>>}\\[0.1cm]
	Транслирующие системы\\[3.0cm]
\end{center}

\begin{flushright}
	\begin{minipage}{0.45\textwidth}
		\textbf{Работу выполнил студент}\\[3mm]
		группа 43501/3 \hfill Дьячков В.В.\\[5mm]
		\textbf{Работу принял преподаватель}\\[5mm]
		\sign[2cm] \hfill к.т.н., доц. Цыган В.Н. \\[5mm]
	\end{minipage}
\end{flushright}

\vfill

\begin{center}
	Санкт-Петербург\\[0.3cm]
	\the\year
\end{center}
\end{titlepage}

\addtocounter{page}{1}

\tableofcontents
\newpage

\section{Цель работы}

Составить \code{Lex} и \code{yacc}-программу, которая будет осуществлять разбор оператора \code{if} (без \code{else}) языка \code{C} с типом данных -- \code{int}. Условное выражение -- бесскобочное арифметическое выражение, содержащее операторы \code{+ - * /}, где все операции выполняются слева направо. Тело оператора:
\begin{itemize}
	\item или только операторы присваивания вида \code{id =} бесскобочное арифметическое выражение (аналогично условному выражению);
	\item или вложенный оператор \code{if}.
\end{itemize}

\section{Выполнение работы}

\inout{test1}

\section{Выводы}

В процессе выполнения данной работы:

\begin{itemize}
	\item был закреплен на практике навык написания программ на языке лексического разбора \code{Lex} и языке синтаксического анализатора \code{yacc};
	\item реализована собственная программа, позволяющая производить разбор условного оператора \code{if} языка \code{C}, содержащего в теле оператора присваивания или вложенные операторы \code{if}.
\end{itemize}

\end{document}
