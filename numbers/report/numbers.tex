\documentclass[a4paper,14pt]{extarticle}

\usepackage[utf8]{inputenc}
\usepackage[T1]{fontenc}
\usepackage[russian]{babel}
\usepackage{hyperref}
\usepackage{indentfirst}
\usepackage{here}
\usepackage{array}
\usepackage{graphicx}
\usepackage{caption}
\usepackage{subcaption}
\usepackage{chngcntr}
\usepackage{amsmath}
\usepackage{amssymb}
\usepackage[left=2cm,right=2cm,top=2cm,bottom=2cm,bindingoffset=0cm]{geometry}
\usepackage{multicol}
\usepackage{multirow}
\usepackage{titlesec}
\usepackage{listings}
\usepackage{color}
\usepackage{enumitem}
\usepackage{cmap}
\usepackage{pgfplotstable}
\usepackage{listingsutf8}

\definecolor{green}{rgb}{0,0.6,0}
\definecolor{gray}{rgb}{0.5,0.5,0.5}
\definecolor{purple}{rgb}{0.58,0,0.82}

\lstdefinelanguage{none}{}

\lstset{
	language={C},
	inputpath={../src},
	backgroundcolor=\color{white},
	commentstyle=\color{green},
	keywordstyle=\color{blue},
	numberstyle=\color{gray}\scriptsize\ttfamily,
	stringstyle=\color{purple},
	basicstyle=\lst@ifdisplaystyle\footnotesize\fi\ttfamily,
	breakatwhitespace=false,
	breaklines=true,
	captionpos=b,
	keepspaces=true,
	numbers=left,
	numbersep=5pt,
	showspaces=false,
	showstringspaces=false,
	showtabs=false,
	tabsize=4,
	frame=single,
	morekeywords={},
	deletekeywords={},
	morecomment=[l][\color{green}]{\$\$},
	sensitive=true,
	columns=fullflexible,
	inputencoding=utf8/cp1251,
	literate=%
		{~}{{\raise.25ex\hbox{$\mathtt{\sim}$}}}{1}%
		{-}{-}{1}
}

\renewcommand{\le}{\ensuremath{\leqslant}}
\renewcommand{\leq}{\ensuremath{\leqslant}}
\renewcommand{\ge}{\ensuremath{\geqslant}}
\renewcommand{\geq}{\ensuremath{\geqslant}}
\renewcommand{\epsilon}{\ensuremath{\varepsilon}}
\renewcommand{\phi}{\ensuremath{\varphi}}
\renewcommand{\thefigure}{\arabic{figure}}
\newcommand{\code}[1]{\lstinline|#1|}
\newcommand{\caret}{\^{}}
\newcommand{\lex}[1]{
	\lstinputlisting[caption=\code{#1.l}]{#1.l}
}
\newcommand{\yacc}[1]{
	\lstinputlisting[caption=\code{#1.y}]{#1.y}
}
\newcommand{\inout}[1]{ 
	\lstinputlisting[language=,caption=\code{#1.in}]{#1.in}
	\lstinputlisting[language=,caption=\code{#1.out}]{#1.out}
}

\titleformat*{\section}{\large\bfseries} 
\titleformat*{\subsection}{\normalsize\bfseries} 
\titleformat*{\subsubsection}{\normalsize\bfseries} 
\titleformat*{\paragraph}{\normalsize\bfseries} 
\titleformat*{\subparagraph}{\normalsize\bfseries} 

\counterwithin{figure}{section}
\counterwithin{equation}{section}
\counterwithin{table}{section}
\newcommand{\sign}[1][5cm]{\makebox[#1]{\hrulefill}}
\newcommand{\equipollence}{\quad\Leftrightarrow\quad}
\newcommand{\no}[1]{\overline{#1}}
\graphicspath{{../pics/}}
\captionsetup{justification=centering,margin=1cm}
\def\arraystretch{1.3}
\setlength\parindent{5ex}
\titlelabel{\thetitle.\quad}

\setitemize{topsep=0em, itemsep=0em}
\setenumerate{topsep=0em, itemsep=0em}

\begin{document}

\begin{titlepage}
\begin{center}
	Санкт-Петербургский Политехнический Университет Петра Великого\\[0.3cm]
	Институт компьютерных наук и технологий \\[0.3cm]
	Кафедра компьютерных систем и программных технологий\\[4cm]
	
	\textbf{ОТЧЕТ}\\ 
	\textbf{по лабораторной работе}\\[0.5cm]
	\textbf{<<Индивидуальное задание на языке Lex>>}\\[0.1cm]
	Транслирующие системы\\[3.0cm]
\end{center}

\begin{flushright}
	\begin{minipage}{0.45\textwidth}
		\textbf{Работу выполнил студент}\\[3mm]
		группа 43501/3 \hfill Дьячков В.В.\\[5mm]
		\textbf{Работу принял преподаватель}\\[5mm]
		\sign[2cm] \hfill к.т.н., доц. Цыган В.Н. \\[5mm]
	\end{minipage}
\end{flushright}

\vfill

\begin{center}
	Санкт-Петербург\\[0.3cm]
	\the\year
\end{center}
\end{titlepage}

\addtocounter{page}{1}

\tableofcontents
\newpage

\section{Цель работы}

Составить Lex-программу, которая будет осуществлять преобразование вещественных десятичных чисел со знаком и без знака в нормализованную полулогарифмическую форму вида:
\begin{displaymath}
[+|-]\text{<десятичная точка>}\text{<значение цифры мантиссы>}\text{<порядок числа>},
\end{displaymath}
где порядок числа имеет вид: $E+|-\text{<две значащие цифры порядка>}$.

Синтаксически некорректные входные данные должны печататься с соответствующим диагностическим сообщением, например:

\begin{itemize}
	\item \code{-+123.4} -- два знака подряд;
	\item \code{0..000432} -- две точки в одном числе.
\end{itemize}

\section{Выполнение работы}

Для преобразования напишем вспомогательные функции:

\begin{itemize}
	\item \code{void normalize(double *value, int *power)} -- нормализация переданного значения \code{value}, т.е. приведение к диапазону $[0.1, 1)$. При этом в \code{power} записывается порядок числа.
	\item \code{void format(const char* str)} -- функция, принимающая считанный \code{yytext}, содержащий строковое представление десятичного числа, и вызывающая для разобранного числа функцию \code{normalize}. После этого формируется число в нормализованном полулогарифмическом формате и выводится в консоль.
\end{itemize}

Для поиска диагностических сообщений напишем свои шаблоны. При выявлении ошибок будет печататься сообщение, а преобразование происходить не будет. Кроме того, добавим шаблон для подавления посторонних символов.

\lex{numbers}

Скомпилируем Lex-программу и подадим на вход тестовые примеры. Первые 6 примеров являются допустимыми входными последовательностями, в то время последние два являются ошибочными.

\inout{numbers}

Видно, что допустимые входные последовательности были успешно преобразованы в полулогарифмическую форму, в то время как для ошибочных последовательностей были выведены диагностические сообщения.

\section{Выводы}

В процессе выполнения данной работы:

\begin{itemize}
	\item был закреплен на практике навык написания программ на языка лексического разбора \code{Lex};
	\item реализована собственная программа, позволяющая преобразовывать вещественные десятичные числа со знаком и без знака в нормализованную полулогарифмическую форму.
\end{itemize}

\end{document}
